\section{Fazit}\label{fazit}


Das Ziel dieser Bachelorarbeit war es den Prozess des Markieren eines Bildes zu optimieren. Dazu wurden gängige Label-Tools analysiert und in einer Pro- und Contra-Liste ausgewertet. Da keines der Programme effizient für das Erstellen großer Trainingsdatensätze verwendet werden konnte, wurde ein eigener Ansatz gewählt. Um dies zu realisieren, wurden die Grundlagen und das AnnotationTool von Herrn Mario Hoffmann, welches die Vorlage für die Umsetzung des eigenen Ansatzes darstellt, analysiert. Zum besseren Verständnis der geforderten Kriterien durch den Kunden wurden Anwendungsdiagramme erstellt und Anforderungen abgeleitet, welche konzeptionell beschrieben und implementiert wurden.
\\
\noindent
Im Ergebnis zeigt sich ein Programm, welches gegenüber dem händischen Labeln von Bildern einen deutlichen Vorteil bietet. Die Kombination aus Catmull-Rom Spline und der Matrizenmultiplikation, zur Umrechnung von Pixel- in Weltkoordinaten, erwies sich als effektiv.Die Tests ergaben, dass das Labeln mit dem LabelTool deutlich schneller von der Hand geht, als mit den beschriebenen Label-Tools. Der große Vorteil des LabelTools liegt darin, dass beide Schienen mit einem Punkt markiert werden können und das Schienenbett sich automatisch aus den jeweils gegenüberliegenden Ränder-Punkten bilden lässt. Die Ergebnisse zeigten, dass man in einem normalen Szenario mit 2-3 Schienen pro Bild ca. 1:35 Minute für das Markieren eines Bildes benötigt. Die analysierten Programme benötigen für diesen Prozess zwischen 15-20 Minuten. Die semi-autogenerierten Mittelpunkt-Label zeigten größtenteils gute Ergebnisse, jedoch gab es durchweg keine perfekten Bilder, da die Punkte aus denen die Label berechnet wurden, immer einen festen Abstand zueinander hatten, weshalb die Schienen oft nicht genau abgebildet werden konnten.  Besonders wenn sich keine Bodenunebenheiten in der zu markierenden Szene befinden, funktioniert das LabelTool gut. Andernfalls verschiebt sich die Perspektive durch die Veränderung des Nick-Winkels der Kamera, sodass die Umrechnung von Pixel- in Weltkoordinaten nicht mehr korrekt ist. Dieser Umstand kann umgangen werden, indem die Kamera-Matrix aus dem Programm heraus eingestellt werden kann. An dieser Vorgehensweise soll in Zukunft weiter geforscht werden, sodass im Ausblick das Programm weiter optimiert und die Zuverlässigkeit der Ergebnisse verbessert wird.

