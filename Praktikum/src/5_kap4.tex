\section{Weiterführende langfristige Aufgaben }
\label{sec:Weiterführende langfristige Aufgaben}

Nach der Durchführung meines ersten eigenen Projekte und der Festigung von Bindungen mit Kollegen über verschiedene Bereiche, wurde ich als vollwertiges Mitglied und Ansprechpartner in vielerlei Hinsicht angesehen. Dadurch entwickelten sich für mich eigene Aufgabengebiete, welche sich aus der Situation entwickelten, dass ich den Ruf als “Mädchen für alles” weg hatte. 

\subsection{Persönlicher Support der Führungsebene}
\label{sec:Persönlicher Support der Führungsebene}

Bald war in der DKB bekannt, dass man bei Problemen bei mir schnell eine Lösung finden konnte, da ich neben dem Tagesgeschäft noch Kapazitäten hatte, welche dafür genutzt werden konnten. So kamen Kollegen mit einem breiten Spektrum an Problemen, welche sowohl Hardware als auch Software umfassten, zu mir und suchten Hilfe. Die häufigsten Probleme traten bei Macbooks der Grafikdesigner in der Taubenstraße auf. Dort war ich eine Zeit regelmäßig zugegen, da diese nach Updates oft Probleme bereiteten. In der Regel reichte es die Macbooks zurückzusetzen oder das Update zu deinstallieren. In dieser Zeit lernte ich besonders den Umgang mit Kollegen, Problemen und besonders die Dankbarkeit der Mitarbeiter kennen, wenn das Problem schneller behoben wurde, als gedacht. In der weiteren Zeit kamen immer wieder Kollegen mit kleineren Problemen auf mich zu, welche ich versuchte zu lösen.

\subsection{Automation von alltäglichen Aufgaben}
\label{sec:Automation von alltäglichen Aufgaben}

Besonders intensiv beschäftigte ich mich nach meinen ersten größeren Projekten mit der Automation von alltäglichen Aufgaben im Büro und versucht mit Hilfe der Programmiersprache Python Abläufe zu beschleunigen und Prozesse zu vereinfachen. 
Mein erstes kleineres Projekt in diese Richtung war der Abgleich zweier Datenbanken, welche in Form von Excel-Tabellen existierten, und die Korrektur von falschen Einträgen. Diese Notwendigkeit entstand daraus, dass man die existierende Mobilfunk-Datenbank komplett erneuern wollte. In dieser Datenbank wurden die iPhones, iPads und Notebooks registriert, welche an Mitarbeiter ausgegeben wurden, sodass diese Datenbank doch sehr wichtig für das Unternehmen war. Eine Überarbeitung war notwendig, da die Datenbank über die Jahre mitgewachsen und nicht wirklich optimiert wurde, sodass die Grundlage, welche auf Microsoft-Access basierte, nicht mehr zeitgemäß war und es lange Ladezeiten gab. In der ersten Analyse fiel auf, dass die Datenbank viele inkonsistente Einträge aufwies, welche glatt gezogen werden mussten. Dazu konnte ein Excel-Export gezogen werden, welche dann auf Inkonsistenzen oder falsche Einträge überprüft werden konnte. Um dies zu Überprüfen verwendete ich eine andere Datenbank (Excel-Tabelle) in der Namen und Hardwaren ebenfalls vorhanden waren und welche nur von unserem Team geführt wurde. In dieser Datenbank wurde noch einmal explizit von unserem Team der Name, die Hardware und die korrespondierende Handynummer eingetragen und geführt. So schrieb ich ein Programm, welches jeweils zwei die Excel-Tabellen miteinander verglich, falsche Einträge und Unterschiede aufzeigte. Dies war möglich, da die Namen in beiden Tabellen gleich waren. Durch meine Hilfe konnten schnell Unstimmigkeiten in den jeweiligen Tabellen erkannt und behoben werden, sodass die neue Datenbank nur mit korrekten und aktuellen Einträgen gefüllt werden konnte. Im weiteren Verlauf versuchte ich mich an diversen Automatisierungen, welche mal gut und mal weniger gut funktionierten. So versuchte ich die Erstellung eines Übergabeprotokolls zu automatisieren. Bis dato sah der Prozess so aus, dass man den jeweiligen Mitarbeiter händnisch in der internen Datenbank suchen musste und diesen danach händnisch in das Übergabeprotojoll einzutragen. Danach mussten die Eckdaten zu der übersendeten Hardware angegeben werden. Mit meinem Programm versuchte ich diese Schritte weitestgehend zu automatisieren. Dazu musste der Benutzer lediglich den Namen als Input eingeben und das Programm öffnete automatisch die entsprechende Seite und suchte nach dem Namen in der Datenbank. Das war möglich, da der Name einfach in der URL ergänzt werden konnte und so einfach gefunden wurde. Durch erste Erfahrungen in Picture-to-Text konnte ich nun einen Screenshot machen und automatisch den Namen und Adresse herausfiltern. Da ich in einem vorherigen Projekt schon automatisierte Übergabeprotokolle erstellte, konnte ich diese Funktion nutzen um nun den Namen und die jeweiligen Hadrware, welche einfach im Programm angegeben werden konnte, einzusetzen. Weiterhin programmierte ich ein kleines Tool, welches zu einem bestimmten Namen einen individuellen 4-stelligen Code erstellte. Diese Codes waren notwendig, da die Telefone (iPhone 8) mit einem Sperrcode an die Mitarbeiter verschickt wurden, welcher ihnen ausgehändigt wurde, sobald das unterschriebene Übergabeprotokoll abgegeben wurde. Bis dato wurde immer der selbe Code 1122 verwendet, welcher jedoch irgendwann unter den Kollegen bekannt war, sodass sie die Telefone entsperren konnten, ohne das Übergabeprotokoll abgegeben zu haben. Man konnte in dem Programm den Namen eingeben und es erstellte einen reproduzierbaren individuellen Code, welcher aus den ersten Buchstaben des Vor- und Nachnamen erstellt wurde und nicht einfach herausgefunden werden konnte. Das Tool wird noch heute eingesetzt und sicherte, dass die Kollegen ihre Übergabeprotokolle wirklich unterschreiben und abgeben mussten. Der Code dazu befindet sich im Git-Repository.

\subsection{Pflege der Datenbank}
\label{sec:Pflege der Datenbank}

Neben meinen Hauptaufgaben versuchte ich nebenbei die vorhandene Datenbank zu pflegen und eventuell falsche Einträge zu korrigieren und ein neues relationales Datenbankmodell zu entwerfen, da bei der existierenden Datenbank weder auf die dritte Normalform, noch auf andere Datenbanken-Schemata geachtet wurde, weshalb diese viele Redundanzen und Datenanomalien aufwies. Dies diente jedoch nur dem Training, da ein komplettes Team an einer komplett neuen Datenbank arbeitete. Da es sich hier um eine Nebenaufgabe handelte, welche mir eher zu Schulungszwecken diente, bekam diese Aufgabe im weiteren Verlauf immer weniger Aufmerksamkeit, sodass mein Fokus nach einem Monat anderen Dingen galt. Jedoch konnte ich in der Zeit mein Wissen aus dem Studium erweitern und dies auch auf eigene Datenbanken anwenden, da ich nebenbei das Modul Datenbanken belegte und ein eigenes Abschlussprojekt umgesetzt habe. 