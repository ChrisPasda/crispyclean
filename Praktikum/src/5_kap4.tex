\section{Weiterführende Aufgaben }
\label{sec:Weiterführende Aufgaben}

Nach der Durchführung meines ersten eigenen Projektes und der Festigung von Bindungen mit Kollegen über verschiedene Bereiche, wurde ich als vollwertiges Mitglied und Ansprechpartner in vielerlei Hinsicht angesehen. Dadurch entwickelten sich für mich eigene Aufgabengebiete, welche sich aus der Situation entwickelten, dass ich den Ruf als “Mädchen für alles” weg hatte. 

\subsection{Persönlicher Support der Führungsebene}
\label{sec:WPersönlicher Support der Führungsebene}

Bald war in der DKB bekannt, dass man bei Problemen bei mir schnell eine Lösung finden konnte, da ich neben dem Tagesgeschäft noch Kapazitäten hatte, welche dafür genutzt werden konnten. So kamen Kollegen mit einem breiten Spektrum an Problemen, welche sowohl Hardware als auch Software umfassten, zu mir und suchten Hilfe. Die häufigsten Probleme traten bei Macbooks der Grafikdesigner in der Taubenstraße auf. Dort war ich eine Zeit regelmäßig zugegen, da diese nach Updates oft Probleme bereiteten. In der Regel reichte es die Macbooks zurückzusetzen oder das Update zu deinstallieren. In dieser Zeit lernte ich besonders den Umgang mit Kollegen, Problemen und besonders die Dankbarkeit der Mitarbeiter kennen, wenn das Problem schneller behoben wurde, als gedacht. In der weiteren Zeit kamen immer wieder Kollegen mit kleineren Problemen auf mich zu, welche ich versuchte zu lösen.

\subsection{Automation von alltäglichen Aufgaben}
\label{sec:Automation von alltäglichen Aufgaben}

Besonders intensiv beschäftigte ich mich nach meinem ersten großen Projekt mit der Automation von alltäglichen Aufgaben im Büro und versucht mit Hilfe der Programmiersprache Python Abläufe zu beschleunigen und Prozesse zu vereinfachen. 
Mein erstes kleineres Projekt in diese Richtung war der Abgleich zweier Datenbanken, welche in Form von Excel-Tabellen existierten, und die Korrektur von falschen Einträgen. Diese Notwendigkeit entstand daraus, dass man die existierende Mobilfunk-Datenbank komplett erneuern wollte. In dieser Datenbank wurden die iPhones, iPads und Notebooks registriert, welche an Mitarbeiter ausgegeben wurden, sodass diese Datenbank doch sehr wichtig für das Unternehmen war. Eine Überarbeitung war notwendig, da die Datenbank über die Jahre mitgewachsen und nicht wirklich optimiert wurde, sodass die Grundlage, weaalche auf Microsoft-Access basierte, nicht mehr zeitgemäß war und es lange Ladezeiten gab. In der ersten Analyse fiel auf, dass die jeweiligen Datenbanken oft falsche Einträge aufwiesen, welche glatt gezogen werden mussten. So schrieb ich ein Programm, welches jeweils zwei Excel-Tabellen miteinander verglich und Unterschiede aufzeigte. Dies war möglich, da der Name in beiden Tabellen gleich war. Durch meine Hilfe konnten schnell Unstimmigkeiten in den jeweiligen Tabellen erkannt und behoben werden, sodass die neue Datenbank nur mit korrekten und aktuellen Einträgen gefüllt werden konnte. Danach analysierte ich Abläufe und befragte Kollegen, bei welchen Prozessen ihrer Meinung nach viel Zeit verloren ging und optimiert werden konnten. So fand ich heraus, dass neu-gelieferte Hardware händnisch eingescannt und dadurch registriert werden musste. An den Verpackungen von iPhones, iPads und Laptops befanden sich außen Barcodes(repräsentativ für die jeweilige Imei-Nummer der Hardware), welche mittels eines Scanners eingescannt und in einer Excel-Tabelle festgehalten wurden. Dieser Prozess war sehr aufwändig und zeitraubend, da die Barcodes sehr klein und schwer genau zu Treffen waren. So entschied ich mich mit Python und der Bibliothek Tesseract eine Bilderkennung zu programmieren, welche die Barcodes anhand eines Bildes in einer Excel-Tabelle speichern konnte, sodass lediglich ein Foto von den Barcodes eines jeden Kartons gemacht werden mussten. Um dies umzusetzen experimentierte ich in den ersten Wochen viel mit der Bilderkennung von der Vorverarbeitung, damit vernünftige Resultate erzielt werden konnten. Es stellte sich heraus, dass es doch sehr schwer war ein verlässliches Ergebnis zu erzielen, bei dem wirklich jede Imei korrekt erkannt und in der Excel-Tabelle gespeichert werden konnte. Da dieser Prozess eine einhundertprozentige Zuverlässigkeit gewährleisten musste, hat schaffte es das Script trotzdem 99\%iger Erkennungsrate es nicht eingesetzt zu werden, da es immer wieder vorkam, dass eine Imei nicht korrekt erkannt wurde. Während dieses kleinen Projektes lernte ich, dass es bei vielen Prozessen ungemein wichtig ist, dass der geschriebene Code verlässlich ist und dass man die Verantwortung trägt, wenn dabei Probleme auftreten. Des weiteren verbesserte sich mein Umgang mit Python und der automatischen Bilderkennung. Der Code zu dem Script befindet sich auf dem Git.

\subsection{Pflege der Datenbank}
\label{sec:Pflege der Datenbank}

Neben meinen Hauptaufgaben versuchte ich nebenbei die vorhandene Datenbank zu pflegen und eventuell falsche Einträge zu korrigieren und ein neues relationales Datenbankmodell zu entwerfen, da bei der existierenden Datenbank weder auf die dritte Normalform, noch auf andere Datenbanken-Schemata geachtet wurde, weshalb diese viele Redundanzen und Datenanomalien aufwies. Dies diente jedoch nur dem Training, da ein komplettes Team an einer komplett neuen Datenbank arbeitete. Da es sich hier um eine Nebenaufgabe handelte, welche mir eher zu Schulungszwecken diente, bekam diese Aufgabe im weiteren Verlauf immer weniger Aufmerksamkeit, sodass meine Aufmerksamkeit nach einem Monat anderen Dingen galt. Jedoch konnte ich in der Zeit mein Wissen aus dem Studien erweitern und dies auch auf eigene Datenbanken anwenden, da ich nebenbei das Modul Datenbanken belegte und ein eigenes Abschlussprojekt umgesetzt habe. 