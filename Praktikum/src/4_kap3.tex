\section{Einführung in das erste eigene Projekt}
\label{sec:Einführung in das erste eigene Projekt}

Nach der Einarbeitungszeit und dem Kennenlernen der Hard- und Software wurde mir nach ca. 3 Monaten mein erstes eigenes Projekt überlassen. In diesem ging es um die ganzheitliche Planung, Organisation und Umsetzung von innerbetrieblichen Schulungen. Bei diesen Schulungen handelte es sich um SQL-, Azure-, SAP- und weitere spezielle Mitarbeiter-Schulungen. Diese Schulungen wurden regelmäßig von der DKB-Management-School geplant und durchgeführt. Mein Aufgabenbereich erstreckte sich von der Annahme der Tickets für Schulungen, dem Buchung der Schulungs-Hardware, dem Verschicken der richtigen Hardware und der Organisation des Aufbaus beim Kunden. Im weiteren Verlauf versuchte ich den Prozess des Buchens von Hardware zu verbessern, indem ich eine eigene Buchungsseite programmiert habe, welche mit einer Datenbank kommuniziert, die auf einem Server bei uns im Büro stand. Die Seite sollte schlicht und übersichtlich gehalten werden und wurde in Python geschrieben. Um abrufen zu können, welche Laptops gebucht wurden, habe ich im weiteren Verlauf ein kleines Programm mit Benutzeroberfläche geschrieben, welche mit der Datenbank kommunizieren und gebuchte Rechner abrufen konnte. Jeder Buchung wurde das entsprechende Ticket zugeordnet und es konnte ein Blatt ausgedruckt werden, welches die genauen Hardwareanforderungen für die jeweilige Schulung enthielt. Nach einer Buchung wurde automatisch einen E-Mail an meine Adresse mit der gebuchten Hardware geschickt. So konnte der Prozess des Buchens und der Vorbereitung einer Schulung deutlich schneller durchgeführt werden. Außerdem optimierte ich die Erstellung des entsprechenden Übergabeprotokolls, welches die Korrektheit der Lieferung bestätigen sollte. Dazu implementiere ich in der Datenbank die Funktion direkt per Knopfruck ein Übergabeprotokoll für die gebuchte Hardware zu erstellen. Während der ganzen Zeit stand mir ein Kollege zur Seite, welcher sich besonders gut mit Netzwerken und Datenbanken auskennt, sodass ich immer einen Ansprechpartner hatte. Nach der erfolgreichen Buchung und Vorbereitung einer Schulung, musste ich den Aufbau organisieren. Dazu wurde eine externe Firma beauftragt und dabei handelte es sich um die Schindler AG, welche ich vor jeder Schulung mit einer Woche vorlauf telefonisch kontaktierte und einen Aufbau koordinierte. Wenn Probleme mit den Terminen bei Schindler auftraten, dann bin ich persönlich zum jeweiligen Standort der Schulung gefahren und habe die Hardware entsprechend aufgebaut. Dabei mussten alle Laptops entsprechend verkabelt, hochgefahren und betriebsbereit konfiguriert werden. Abschließend wurde ein Netzwerk aufgebaut, welches eine Kommunikation mit dem Netz der DKB gewährleistete. Dazu wurde ein oder mehrere Switches mit dem Netzwerk verbunden, indem die im Büro befindlichen Steckdosen für das Hauseigene Netzwerk verwendet wurden. Danach wurden die Laptops mit dem Switch verbunden, sodass alle auf das Netzwerk zugreifen konnten. Bei dem Netzwerk der DKB handelte es sich um zwei voneinander gekoppelte Netzwerke, welche in ein Bank- und ein Servicenetz gegliedert wurden. In einem großen Vorhanden mit dem Namen “Netz 4.0”, sollten diese Netze miteinander fusioniert werden, was jedoch nach drei Anläufen bis heute nicht gelang. So musste meist ein Netzwerk für das Bank-Netz hergestellt werden, sodass die Schulungsmitglieder zugriff auf die SQL- oder Bank-Server hatten. Fehlerhafte Hardware musste entsprechend ausgetauscht oder repariert werden. Bei der Zusammensetzung der Hardware musste darauf geachtet werden, dass von bestimmten anfälligen Hadrwarekomponenten mehr als gefordert verschickt wurden, sodass kaputte Komponenten kompensiert werden konnten. Mit der Zeit bekam ich ein Gefühl dafür, welche Hardware oft kaputt geht. So waren es in der Regel Kabel und Netzteile, welche in der Zeit am Öftesten kaputt gingen. Öfter mussten auch Switches ausgetauscht werden, was jedoch kein Problem darstellte, da ich meist zwei Switches mehr als gefordert versendete. Spezielle Hardwareanforderungen der Kunden mussten entsprechend von mir beschafft und umgesetzt werden. Dabei ging es in der Regel mehr um Softwareanforderungen, als um Hardware. Da die Laptops entsprechend gehärtet wurden, sodass keine externen Programme einfach aufgespielt werden konnten, stellte dies oft ein Problem dar. Dieses konnte nur gelöst werden, indem ein komplett unabhängiges Betriebssystem installiert wurde, welches ein Aufspielen externer Software erlaubte. Nach Beendigung der Schulungen musste dieser Prozess wieder rückgängig gemacht werden und das alte Betriebssystem wieder aufgespielt werden. Desweiteren musste ich entsprechende Teilnehmer-Accounts auf den Laptops einrichten, welche eine zeitliche Begrenzung hatten und jemandem Mitarbeiter entsprechend zugeordnet werden konnten. 
In diesem Projekt lernte ich vor allem die Kommunikation und die Zusammenarbeit mit verschiedensten Kollegen kennen und konnte meine Fähigkeiten im Bereich der Programmierung von Datenbanken und Web-Anwendungen verbessern und trainieren.Weiterhin wurde mein Organisationsfähigkeit und das Verständnis der Umsetzung von Hard- und Softwareanforderungen geschult. 

\subsection{Die Buchungsseite für Schulungshardware}
\label{Die Buchungsseite für Schulungshardware}



